\documentclass[unicode,11pt,a4paper,oneside,numbers=endperiod,openany]{scrartcl}
\newcommand\tab[1][0.5cm]{\hspace*{#1}}
\usepackage{array}
\usepackage{multirow}
\usepackage{graphicx}
\usepackage[utf8]{inputenc}
\usepackage{listings}
\usepackage{xcolor}
\usepackage{seqsplit}
\usepackage{float}
\usepackage{booktabs}
\usepackage{subcaption}
\usepackage{adjustbox}
\usepackage{listings}
%New colors defined below
\definecolor{codegreen}{rgb}{0,0.6,0}
\definecolor{codegray}{rgb}{0.5,0.5,0.5}
\definecolor{codepurple}{rgb}{0.58,0,0.82}
\definecolor{backcolour}{rgb}{0.98,0.98,0.98}
%Code listing style named "mystyle"
\lstdefinestyle{mystyle}{
  backgroundcolor=\color{backcolour}, commentstyle=\color{codegreen},
  keywordstyle=\color{magenta},
  numberstyle=\tiny\color{codegray},
  stringstyle=\color{codepurple},
  basicstyle=\ttfamily\footnotesize,
  breakatwhitespace=false,         
  breaklines=true,                 
  captionpos=b,                    
  keepspaces=true,                 
  numbers=left,                    
  numbersep=5pt,                  
  showspaces=false,                
  showstringspaces=false,
  showtabs=false,                  
  tabsize=2,
  numbers=none
}
\lstdefinestyle{base}{
  language=C,
  emptylines=1,
  breaklines=true,
  basicstyle=\ttfamily\color{black},
  moredelim=**[is][\color{red}]{@}{@},
}
\lstset{style=mystyle}
\newcommand\MyBox[2]{
  \fbox{\lower0.75cm
    \vbox to 1.7cm{\vfil
      \hbox to 1.7cm{\hfil\parbox{1.4cm}{#1\\#2}\hfil}
      \vfil}%
  }%
}
\input{assignment.sty}

\usepackage{subcaption}

\begin{document}


\setassignment

\serieheader{Particle Methods}{2023}{Student: Filippo Casari}{}{Report for Assignment 2}{}
\newline
\section*{Introduction}
To achieve the goal of the assignment I implemented 3 versions:
\begin{itemize}
  \item \textbf{MATLAB 2D}: Using just lists and the non-optimized version of the algorithm. Complexity: $O(n^2)$
  \item \textbf{Python 2D}: In the beginning I used lists of objects and then I switched to dictionaries for better performances. Complexity: $O(n)$ using cell lists. 
  \item \textbf{Python 3D}: Same as the previous version but in a 3D environment. Complexity: $O(n)$
\end{itemize}
Unfortunately, I found the best parameters that can lead to similar  results for Lotka-Volterra equations. I am going to list those parameters later. 
\section*{Initial Conditions}
The rabbits and the wolves are distributed randomly within the environment as shown below (example). I decided to not set fixed seed for random functions
to have every time a different initial condition.
\begin{figure}[H]
  \centering
  \includegraphics[width=0.5\textwidth]{output_main/Initial_condition.png}
  \caption{Initial conditions}
\end{figure}




\section*{Point A}
In both MATLAB and Python implementations after very few hundred iterations and with the given parameters the wolves die. In contrast what I expected, the wolves death rate maybe is high probably because the rabbit birth rate is not high enough.\\
By running main.py I draw the following plots:
\begin{itemize}
  \item \textbf{Simulation}
  \item \textbf{Number of rabbits and wolves over iterations}
  \item \textbf{Rabbits and wolves density}
\end{itemize}
\begin{figure}[H]
  \centering
  \includegraphics[width=0.5\textwidth]{output_main/PointA.png}
  \caption{PointA;2D; Number of rabbits and wolves over iterations}
\end{figure}
\begin{figure}[H]
  \centering
  \includegraphics[width=0.6\textwidth]{output3D/3DInitCondition.png}
  \caption{RT visualization}
\end{figure}
In the 3D plot the number of rabbits grows exponentially as well with more than 8000 rabbits only in the very first iterations. 
\begin{figure}[H]
  \centering
  \includegraphics[width=0.6\textwidth]{output3D/secondPlotPointA.png}
  \caption{Environment visualization after just 200 iters} 
\end{figure}

\end{document}
